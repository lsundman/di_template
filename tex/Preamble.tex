% Peruspaket
\usepackage{graphicx}
\usepackage{amsmath}

% Paket som skapar PDF/A
\usepackage[a-1b,mathxmp]{pdfx}
% Viktigt för PDF/A om du använder EPS-bilder, tar bort färger som int funkar
\usepackage{epstopdf}
\epstopdfDeclareGraphicsRule{.eps}{pdf}{.pdf}{%
    epstopdf --gsopt=-sColorConversionStrategy=Gray --gsopt=-dProcessColorModel=/DeviceGray --gsopt=-dCompatibilityLevel=1.4 #1 --outfile=\OutputFile}

% Nyare sätt att sätta språket
\usepackage{polyglossia}
\setdefaultlanguage{swedish}

% Custom typsnitt
% https://tex.stackexchange.com/questions/59702/suggest-a-nice-font-family-for-my-basic-latex-template-text-and-math
\usepackage{unicode-math}
%\setmainfont{EB Garamond}
%\setmathfont{Garamond-Math}

% Jättebra paket som låter en ha subfigures
\usepackage[labelformat=simple]{subcaption}
\renewcommand\thesubfigure{\,(\alph{subfigure})}
\renewcommand\thesubtable{\,(\alph{subfigure})}

% För titelsidan, några extra makron
\usepackage{titling}

% Biblatex - enklare att använda än plain bibtex, stöder utf-8 etc
\usepackage[
	backend=biber,
	style=ieee,
	eprint=false,
	isbn=false,
	maxnames=2,
	minnames=1
]{biblatex}
\addbibresource{Diplomarbete.bib}

% För ordlistor
\usepackage{glossaries-extra}

% Skapar ordlistor
\makeglossaries

% Kursivt när förkortning skrivs ut
\renewcommand{\glsfirstlongdefaultfont}[1]{\emph{#1}}
\newabbreviation{html}{HTML}{HyperText Markup Language}


% di_direktiv.pdf: enkelt radavstånd mellan stycken, indrag 0.4 cm
\setlength\parindent{0.4cm}

% di_direkt.pdf: fontstorlek på sektioner
\usepackage{sectsty}
\sectionfont{\large}
\subsectionfont{\normalsize}
\subsubsectionfont{\itshape\normalsize}

% di_direktiv.pdf: sidnummer i hörnet
\usepackage{scrlayer-scrpage}
\ohead*{\pagemark}
