% Metadata för PDF/A
\begin{filecontents*}{\jobname.xmpdata}
	\Title{TitelTitel Titel}
	\Author{Förnamn Efternamn}
	\Language{sv-SE}
	\Keywords{hepp\sep hepp\sep hepp\sep hepp}
	\Publisher{Åbo Akademi}
\end{filecontents*}

%%%%%%%%%%%%%%%%%%%%%%%%%%%%%%%%

\documentclass[a4paper,12pt,hidelinks]{article}
\usepackage[left=4.5cm, right=2.5cm, top=2.5cm, bottom=2.5cm]{geometry}

% Peruspaket
\usepackage{graphicx}
\usepackage{amsmath}

% Paket som skapar PDF/A
\usepackage[a-1b,mathxmp]{pdfx}
% Viktigt för PDF/A om du använder EPS-bilder, tar bort färger som int funkar
\usepackage{epstopdf}
\epstopdfDeclareGraphicsRule{.eps}{pdf}{.pdf}{%
    epstopdf --gsopt=-sColorConversionStrategy=Gray --gsopt=-dProcessColorModel=/DeviceGray --gsopt=-dCompatibilityLevel=1.4 #1 --outfile=\OutputFile}

% Nyare sätt att sätta språket
\usepackage{polyglossia}
\setdefaultlanguage{swedish}

% Custom typsnitt
% https://tex.stackexchange.com/questions/59702/suggest-a-nice-font-family-for-my-basic-latex-template-text-and-math
\usepackage{unicode-math}
%\setmainfont{EB Garamond}
%\setmathfont{Garamond-Math}

% Jättebra paket som låter en ha subfigures
\usepackage[labelformat=simple]{subcaption}
\renewcommand\thesubfigure{\,(\alph{subfigure})}
\renewcommand\thesubtable{\,(\alph{subfigure})}

% För titelsidan, några extra makron
\usepackage{titling}

% Biblatex - enklare att använda än plain bibtex, stöder utf-8 etc
\usepackage[
	backend=biber,
	style=ieee,
	eprint=false,
	isbn=false,
	maxnames=2,
	minnames=1
]{biblatex}
\addbibresource{Diplomarbete.bib}

% För ordlistor
\usepackage{glossaries-extra}

% Skapar ordlistor
\makeglossaries

% Kursivt när förkortning skrivs ut
\renewcommand{\glsfirstlongdefaultfont}[1]{\emph{#1}}
\newabbreviation{html}{HTML}{HyperText Markup Language}


% di_direktiv.pdf: enkelt radavstånd mellan stycken, indrag 0.4 cm
\setlength\parindent{0.4cm}

% di_direkt.pdf: fontstorlek på sektioner
\usepackage{sectsty}
\sectionfont{\large}
\subsectionfont{\normalsize}
\subsubsectionfont{\itshape\normalsize}

% di_direktiv.pdf: sidnummer i hörnet
\usepackage{scrlayer-scrpage}
\ohead*{\pagemark}


% Används av titelsidan
\author{TitelTitel Titel}
\date{\today}
\title{Förnamn Efternamn}

\begin{document}

% För att matcha di_direktiv.pdf
\pagenumbering{roman}

\thispagestyle{empty}
\vspace*{3cm}

\begin{center}
	\begin{minipage}{0.6\textwidth}
		\centering
		{\Large\textbf{\MakeUppercase{\thetitle}}} \\
		\vspace{1em}\theauthor
	\end{minipage}
\end{center}

\vfill

\begin{flushright}
	Diplomarbete i datateknik \\
	Handledare: Marina Waldén och Annamari Soini \\
	Fakulteten för naturvetenskaper och teknik \\
	Åbo Akademi \\
	Juni, 2022
\end{flushright}


\setcounter{page}{1}

% Abstrakt
% \include{Sammanfattning}

\tableofcontents
\newpage

\printglossary{}
\newpage

% Vanliga sidnummer efter förorden etc
\pagenumbering{arabic}
\setcounter{page}{1}

% Resten av texten med lite stretchat radavstånd
\renewcommand{\baselinestretch}{1.3}\normalsize

% Genom att sätta TEXINPUTS i .latexmkrc kan vi här includa filer utan att ha med hela pathen

\section{Inledning}

\gls{html} är ett märkspråk.\cite{Eklund2021}

% Bakgrund här ...
% \include{Problembeskrivning}
% Avhandling ...
% \include{Resultat}
% \include{Diskussion}
% \include{Slutord}

% Ändrad titel för att matcha di_direktiv.pdf, bibintoc ger den i
\printbibliography[heading=bibintoc]{}

\end{document}
